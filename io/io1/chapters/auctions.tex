\section{Introduction}

Auctions are an important part of economic research because of their features and applications: they rely on a simple game, with well-specified rules (= less assumptions needed), they can be observed directly, they are used in a lot of settings and are very diverse. All these elements make auctions markets particularly interesting for IO research.

First, we need to define four types of auctions based on the type of bidding and the payment rules: \begin{enumerate}
\item First-price sealed-bid auctions: No agent sees the opponents' bids. Highest bidder wins and pays their bid. Losers get nothing.
\item Second-price sealed-bid auctions: No agent sees the opponents' bids. Highest bidder wins and pays second highest bid. Losers get nothing.
\item English auction: Price goes up until only one player is willing to pay the price. Effectively pays the second highest valuation (in eqm).
\item Dutch auction: Price goes down until one player is willing to pay the price. Effectively pays the highest valuation (in eqm).
\end{enumerate} 

Dutch auctions are not very common and most of the empirical work is centered on the three other types (depending on the situation). Theoretically, we consider FPSB and Dutch auctions to be equivalent (same equilibrium strategies). The same holds between SPSB and English auctions. 

Not covered in this chapter are multi-unit and combinatorial auctions, which both usually ask for a lot more computation power for empirical work.

In the empirical IO literature, the goals of studying auction markets are to (1) describe and elicit, among others, how agents bid, their valuations for the goods, the difference with the models, or the presence of collusion; but also (2) to identify optimal rules given the setting. Researchers typically use two approaches to do this:\begin{itemize}
\item Reduced-form: to test assumptions or theoretical predictions, make inferences about behavior or the bidding environment.
\item Structural: assuming theory holds, to estimate the primitives of the model (distribution of private values, affiliation, etc.)
\end{itemize}

\section{Private Value Auctions}



\section{Common Value Auctions}



\section{Estimation of Auction Models}



\section{Asker (2010)}

\subsection{Summary}

\subsubsection{Background}

Collusion in an English auction can yield lower winning bids (thus lower auction revenues). To see that, consider three bidders with values 10, 8, and 5 in an English auction, such that in a competitive outcome, the price would be 8. If bidders 1 and 2 collude, they can bring the price to 5.  Note that it needs not to happen as when bidders 1 and 3 collude, the price is still 8. The research questions are thus: how do bidding rings work in practice? and how do they affect market outcomes?

To study this: look at bidding ring in the stamps auction market! Eleven dealers are part of a ``ring'' (subset of all bidders). They are organized in two periods: knockout stage and target auction. The first is to determine who from the ring will get the object if the ring wins, and the price at which they stop (with payments), while the second is the actual auction.

\subsubsection{Model}

Structural approach: design of a IPV-style model, with two types of bidders (because weaker bidders in data) and focus only on auctions with two bidders in the knockout phase!

\subsubsection{Data}

Complete record of ring's activity. Observation unit is an auction, observed variables are bidders in the knockout, amount bid, side payments, price of target auction etc.

\subsubsection{Assumptions}



\subsubsection{Results}

Bidding rings can introduce inefficiencies in auctions, but in that case, effect is small. Asymmetry of bidders is a big weakness of maintaining rings. Other bidders (outside the ring) are also affected by the ring.