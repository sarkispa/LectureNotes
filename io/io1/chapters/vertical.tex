\section{Introduction}

Manufacturers rarely supply final consumers directly but are usually vertically separated by one or more intermediaries. In this particular type of setting, we often refer to the manufacturer as the upstream firm, and the intermediary as the downstream firm. This relationship is the same as in a typical market, considering the downstream firm as the customer and the upstream firm as the producer, thus leading to the same topics as usual: endogenous pricing, price discrimination, etc.

The main addition to the usual models is that now the downstream firm is not the end consumer, but rather an agent that will serve the end consumers, thus doing its own share of pricing, advertising, etc. Because these activities will affect the consumers, the upstream firm has incentives to control the downstream firms in some way, we call this ``vertical control''. There are several types of vertical restraints used by firm:\begin{itemize}
\item Exclusive territories: a single retailer is assigned to a ``territory'' (geographical or not) and has monopoly rights over the area.
\item Exclusive dealings: a retailer that chooses the upstream firm cannot sell nor carry any of the competitors' goods.
\item Full-line forcing: a dealer is committed to sell the whole product line of the upstream manufacturer.
\item Resale Price Maintenance (RPM): a dealer commits to retail prices (or a range) that will hold for the product. Equivalently, quantity forcing or rationing will commit the retailer on the quantity side.
\item Contractual arrangements: more flexible agreement between upstream and downstream firms to transfer the product. Profit and revenue sharing are the most common.
\end{itemize}

\section{Theoretical insights}

\subsection{Basic Framework}

Start with a simple model with a homogeneous good with demand given by $p = a - Q$. Moreover, both the upstream and downstream firms are monopolists. The downstream firm has a distribution cost equal to $d$ (the price it pays for the upstream good), while the upstream firm has a marginal cost equal to $c$.

\subsection{Externalities}

Because the downstream firm has a monopoly over retailing, its optimal strategy is to charge the monopoly price for the product. The manufacturer also has a monopoly over the production of the good, thus will charge a monopoly price to the retailer! 

From this example we can draw multiple results: \begin{itemize}
\item The upstream firm earns higher profits than the retailer.
\item The upstream firm would earn even more by selling directly to the market.
\item Total industry profits are lower than vertically integrated profits.
\end{itemize}
These results come from the presence of two markups! This is what we call double marginalization. Going around this issue can be done by including additional terms in the contracts (RPM, quantity forcing, etc.).

\subsection{Downstream Moral Hazard}



\subsection{Interbrand Competition and Legal Issues}



\newpage

\section{Papers}

\subsection{Mortimer (2008)}

\subsubsection{Background}

This paper studies the efficiency improvements linked to vertical contracts in the video rental industry. It follows the contractual innovation of revenue-sharing which became widespread after 1998. The revenue sharing contracts are designed as an upfront fee per tape, then a royalty based on rentals. This comes in contrast to the previous main contract called ``linear-pricing'' where the retailer would just buy the tape upfront and hold it in inventory at whatever price they chose.

\subsubsection{Model}

Straightforward model of vertical market with contracting. The main resulting equations are used as moment equations in the estimation (by GMM).

\subsubsection{Data}

Main observation unit is the store-title-week level, where observed variables include number of transactions (rentals), etc. At more aggregated levels, store-title pairs include average rental price, type of contract, etc. Title data include the type of movie, the box-office, rating etc. and finally the store data include demographics about the area.

\subsubsection{Assumptions}

\begin{itemize}
\item No coordination between stores, even within a chain.
\item No competition between films and film distributors (= monopolistic suppliers).
\end{itemize}

\subsubsection{Results}

Revenue-sharing contracts had positive effects on welfare across the board (consumer surplus, profits, etc.). Other vertical contracts are having a positive effect on profits while decreasing consumer surplus.

\newpage
\subsection{Conlon and Mortimer (2017)}

\subsubsection{Background}

This paper studies the effects of vertical rebates on efficiency (greater retail effort) and foreclosure (of competing products). Using field experiment, consumer choice and retailer behavior. Vending machine industry: AUD (all-units discount) gives discount per-unit, quantity target and facing requirement.

\subsubsection{Model}

Consumer choice model: random-coefficients logit.

Retailer behavior (portfolio and effort): dynamic model of restocking à la Rust (1987).

\subsubsection{Data}

Dataset from field experiment: observation unit is machine-visit, observed variables are quantity vended, price, other facings, etc.

Data from Mars: AUD contractual terms.

\subsubsection{Assumptions}

\begin{itemize}
\item Instruments: no price coefficient because no variation.
\end{itemize}

\subsubsection{Results}

Vertical rebates do increase effort levels for the retailer. 

\newpage
\subsection{Ho, Ho and Mortimer (2012)}

\subsubsection{Background}

This paper studies the effects of bundling as a vertical contract between manufacturers and retailers (= full-line forcing) on welfare. In the video rental industry, the content is very important so need of both supply and demand models (always renewed, etc.).

\subsubsection{Model}

Sort of dynamic nested-logit model (static with decay) with focus on changing choice sets: typical nested-logit with a decay term (months since release fixed effect).

Portfolio choice for the retailer: using moment inequalities, under the assumption that on average portfolio choice is optimal.

Profit model for counterfactuals.

\subsubsection{Data}

Dataset from Rentrak. Observation unit is transaction (store-title), observed variables are distributor, genre, box-office categories, type of release, etc. and type of vertical contract. Additional information on demographics.

\subsubsection{Assumptions}

\begin{itemize}
\item Data and computation limitation: no stockouts, month-level aggregation, nests based on box-office categories (not genre?), monopolies (because don't observe competition).
\item Instruments: average inventory in other stores of same size (for inventory); average number of movies in group and average within group share in other stores of same size (for within group share); no instrument for price (results were the same).
\item Supply side: stores have perfect foresight of demand
\end{itemize}

\subsubsection{Results}



\newpage
\subsection{Crawford et al. (2017)}

\subsubsection{Background}

This paper studies the welfare effects of vertical integration (a type of vertical contract), in terms of reduced double marginalization (positive) and foreclosure (negative). In the multichannel TV market: acquisitions of regional sports networks by multichannel video programming distributors.

\subsubsection{Model}

Viewership: Static time-allocation problem yields value function.

Subscription: Typical logit model including viewership based utility parameters.

Distributor pricing: Static profit maximization given fees, over bundle choice and price.

Affiliate fee bargaining: Nash-in-Nash bargaining.

\subsubsection{Data}

Downstream data: prices, quantities and characteristics for both cable TV and satellite providers.

Viewership data: individual viewership data and aggregate data.

Channel fees and ad revenues: fees (per subscriber) paid to distribute channel; average ad revenue per subscriber.

\subsubsection{Assumptions}



\subsubsection{Results}

