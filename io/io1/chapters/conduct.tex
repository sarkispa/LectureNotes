\section{Introduction}

Antitrust laws are designed with the goal to control how firms attain and maintain their market position, in order to improve consumer welfare, or welfare in a broader sense.

Around the end of the 19th century, people were growing frustrated at the excessive control of monopolies and cartels over markets. This led to the passage of key legislation and the creation of regulatory agencies. Today, the main institutional actors in antitrust are the Department of Justice (DOJ) and the Federal Trade Comission (FTC).

The creation of these agencies show the implicit belief that society cannot rely on private firms to control the market, nor on consumers to organize and fight against market power. But it remains that the goals of antitrust institutions are unclear: are they figthing for consumer welfare? for economic efficiency? What would happen if some consumers win and some lose?

\section{Mergers}

Mergers are the simplest antitrust situation to analyze because there exists a pre and post outcome.

\subsection{Some cases}



\subsection{Merger guidelines}

The Merger Guidelines are a document issued by the DOJ and the FTC (for the first time in 1982, most recently in 2010) with the purpose of clarifying situations where antitrust agencies might impede merger. 

These ``rules'' seek to prevent firms from merging if it would lead to unilateral exercise of market power or to coordinated interactions. They describe tools for merger analysis in two main sections: screening and analysis.

The role of screening is to raise flags in markets where mergers would lead to significantly concentrated markets, where unfair markups could be sustained. Merger analysis is then completed by simulating over the market under merger and not, to compare both outcomes in terms of welfare, prices, markups, etc.

\subsection{Screening}

Screening can be done using one of two tools:

\subsubsection{Using HHI}

The Herfindahl-Hirschman Index, or HHI, is defined as the sum of squared market shares in a market. While previously designed on its own, it can be derived analytically using the Cournot model of competition.

Using it for merger analysis is done by looking at the current HHI in combination with the potential post-merger HHI and their difference. Small changes would then not be challenged compared to bugger changes, but given a change in HHI, more concentrated markets will get challenged more often.

The problem with using the HHI is market definition. In fact, using market shares implies that competitors are clearly defined, whereas in reality, it is not often the case. Moreover, market definition will have a huge impact on the numbers, and thus on what the guidelines have to say about the merger.

\subsubsection{Using UPP}

Using the intuition that a merger might induce a price change, the Upward Pricing Pressure is a theoretical measure of the incentives of raising prices after a potential merger between two firms. By definition it does not require any market definition because it looks at only the two merging firms. However, it assumes away the price responses of potential competitors, uses the pre-merger prices for one of the goods. 

All in all, it is not a perfect measure but can be used as a quick screening process for unilateral mergers.

\subsection{Market definition}

A market is defined as group of products and a geographic area such that a hypothetical profit-maximizing firm (hypothetical monopolist), not subject to price regulation, likely would impose a small but significant and nontransitory increase in price (SSNIP), assuming that the terms of sale of all other products is held constant. A relevant market is no larger (in terms of products and geography) than is needed to satisfy this criteria.

The intuition behind this definition is to use the fact that if a monopolist (one with the most market power) would not increase prices by a significant amount, then no one would.

Usually, this threshold is set at 5\% although it varies across industries and situations.

Then, we need to define market participants which obviously would include the firms that own the products in the marker, but also other firms such as potential entrants (that could enter if only one firm was in the market).

\subsection{Merger Analysis}

As mentioned earlier, the three steps to studying a merger is:\begin{enumerate}
\item Define a market.
\item Screen using HHI or UPP.
\item Simulate the merger (fully or partially).
\end{enumerate}

The last step is the only we have not looked at yet.

\subsubsection{Merger Simulation}

There are two ways of simulating a merger:\begin{itemize}
\item Partial Merger Simulation: where market outcomes are simulated for the merging products only, holding everything else fixed, then the effects are compared to synergies or cost savings.
\item Full Merger Simulation: where market outcomes are simulated using a structural model describing the behavior of all firms in the market.
\end{itemize}

\subsubsection{Potential issues}

Merger analysis typically will not allow for some elements that will bias the analysis in questions. Among others, we find product repositioning, committed entry, possible entry, efficiency gains, exit, etc.

\section{Conlon and Mortimer (2018)}

