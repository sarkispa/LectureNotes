\section{Literature Review}



\section{Covert (2015)}

\subsection{Summary}

The research question in this paper is: ``Do firms learn (in production)?''

Using data on hydraulic fracturing in the Bakken Shale and a model of input choice under technology uncertainty, Covert shows that firms only learned partially, leaving out 40\% of profits in the process.

\subsubsection{Background}

New industry after advances on how to extract shale gas. Almost 1000\% growth in 8 years.

Fracking is pumping a mix of sand, water and chemicals in the ground = choice to make on the ``recipe'' that affects production and costs! But no one knew at the time how to do it = opportunities for learning.

Firms have their own data (private for 6 months) and then get access to other data (after six months).

\subsubsection{Evidence for learning}

Covert looks for three types of learning: (1) is experience (age of firms) correlated with productivity (oil per well drilled)? Which is estimated using a Benkard type of model. (2) is the choice of inputs more profitable over time? which is estimated using ex ante and ex post profits comparison.

\subsubsection{Results}

One of the first empirical analyses of learning behavior in production. Find that firms increased the profit capturing rate from 20\% to 60\%. No experimenting to learn as firms go to more certain input choices. Firms overweight their data compared to other firms' data.

\subsection{Model}

\subsubsection{Production function}

The output is log-log specified as a function of:\begin{itemize}
\item $t$: the number of days of operation of the well.
\item $D$: the number of days of production.
\item $H$: length of the well.
\item $Z$: other topologic controls
\item $\epsilon$: well-specific shock
\item $\nu$: idiosyncratic shock
\end{itemize}

\subsubsection{Profits}

Firm's profits depend on the usual stuff: share, market size, price, and cost of inputs.

\subsubsection{Preferences}

Firms get utility from the mean profit and standard deviation of profits given an input.

\subsubsection{Gaussian Process Regressions}



\subsection{Comments}

\subsubsection{Results}

Firms underutilize sand and water but: are costs measured correctly? If costs are convex, then findings corroborate optimal choice? Overestimate return to production?

\subsubsection{Public policy}

Delayed disclosure of information lowers barriers to entry while leaving rents on the table.

\subsubsection{Experimenting}

Nice framework but conclusion too hasty? Experimenting is too linked with risk, what if even when experimenting firms would choose safe levels of inputs? Or sub-optimal experimenting?

\subsubsection{Other questions}

Risk aversion or myopia?

Prior beliefs are correctly specified?

\subsection{Discussion}

\begin{itemize}
\item \textbf{Research questions:} How do firms learn to use new technologies?
\item \textbf{Goals of the paper:} test models (learning models) and measure an effect (information on production)
\item \textbf{Importance of the paper:} 1st empirical study of learning behavior.
\item \textbf{Theoretical foundations:} Learning the production function (different than learning-by-doing) \begin{itemize}
\item Strengths: 
\item Shortcomings: 
\end{itemize}
\item \textbf{Empirical strategy:} use Gaussian Process Regression (Bayesian technique) to identify learning process of production distribution. Identifies experimenting by looking at ``taste'' for variance. \begin{itemize}
\item Strengths: perfect fit for learning setting of functions rather than points.
\item Shortcomings: taste for variance is not exactly the same as experimenting.
\end{itemize}
\item \textbf{Data:} well-period level (location, length, sand-water combination, firm identity, production, age) + location characteristics (geological variables) + environment variables (oil prices, etc.) + information available at each period in time.
\begin{itemize}
\item Exogenous:
\item Endogenous: sand-water combination, output.
\end{itemize}
\item \textbf{Results:} Firms learn (form 20\% to 60\% of max profits captured); firms underutilize sand and water; firms overweight their own data; firms do not experiment.
\end{itemize}