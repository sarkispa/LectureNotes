\section{Backus, Conlon and Sinkinson (2019)}

\subsection{Discussion}

\begin{itemize}
\item \textbf{Research question:} Does common ownership (overlapping sets of investors) provide incentives that distort competitive behavior?
\item \textbf{Goals of the paper:} test model and measure an effect (common ownership on markups?).
\item \textbf{Importance of the paper:} Growing body of literature in interested in finding a link since enormous rise in common ownership.
\item \textbf{Theoretical foundations:}  Basic IO/Game Theory framework with profit maximizing competing firms, owned by profit maximizing investors. \begin{itemize}
\item Strengths: yields a simple (observable) way to measure common ownership incentives
\item Shortcomings: simplistic; does not treat endogeneity; no dynamics
\end{itemize}
\item \textbf{Empirical strategy:} use a reduced-form model to show changes in ``modified'' concentration (to include profit sharing) across time.\begin{itemize}
\item Strengths: 
\item Shortcomings: no correlation between variables, just presentation over time, no instruments.
\end{itemize}
\item \textbf{Data:} 13-f filings dataset (investor's share in all companies, both control and profit)
\begin{itemize}
\item Exogenous: 
\item Endogenous: changes in these shares (not controlled for)
\end{itemize}
\item \textbf{Results:} Common ownership incentives have increased, even before BlackRock, Vanguard, etc. Common ownership is positively correlated with retail shares (non 13-f shares).
\end{itemize}